%%%%%%%%%%%%%%%%%%%%%%%%%%%%%%%%%%%%%%%%%
% Twenty Seconds Resume/CV
% LaTeX Template
% Version 1.1 (8/1/17)
%
% This template has been downloaded from:
% http://www.LaTeXTemplates.com
%
% Original author:
% Carmine Spagnuolo (cspagnuolo@unisa.it) with major modifications by 
% Vel (vel@LaTeXTemplates.com)
%
% License:
% The MIT License (see included LICENSE file)
%
%%%%%%%%%%%%%%%%%%%%%%%%%%%%%%%%%%%%%%%%%

%----------------------------------------------------------------------------------------
%	PACKAGES AND OTHER DOCUMENT CONFIGURATIONS
%----------------------------------------------------------------------------------------

\documentclass[letterpaper]{twentysecondcv} % a4paper for A4
\makeatletter
\def\blfootnote{\xdef\@thefnmark{}\@footnotetext}
\makeatother

\renewcommand{\emph}[1]{\textbf{\textit{#1}}}

%----------------------------------------------------------------------------------------
%	 PERSONAL INFORMATION
%----------------------------------------------------------------------------------------

% If you don't need one or more of the below, just remove the content leaving the command, e.g. \cvnumberphone{}

\profilepic{foto.jpg} % Profile picture

\cvname{Hubert Kuczwara} % Your name
\cvjobtitle{} % Job title/career

\cvdate{12 września 1989} % Date of birth
%\cvaddress{United Kingdom} % Short address/location, use \newline if more than 1 line is required
\cvnumberphone{(+48) 723 490 304} % Phone number
%\cvsite{http://en.wikipedia.org} % Personal website
\cvmail{hubert.kuczwara@gmail.com} % Email address
\cvlinkedin{www.linkedin.com/in/hubert-kuczwara/}
\cvgithub{www.github.com/HiK121/}
%----------------------------------------------------------------------------------------

\begin{document}
%

%----------------------------------------------------------------------------------------
%	 ABOUT ME
%----------------------------------------------------------------------------------------

%\aboutme{ \aboutme{}

%----------------------------------------------------------------------------------------
%	 SKILLS
%----------------------------------------------------------------------------------------
\hyphenpenalty=10000
% Skill bar section, each skill must have a value between 0 an 6 (float)
%\skills{{BioPython $\bullet$ {C/C++} $\bullet$  \LaTeX $\bullet$ STL/1},{CATIA $\bullet$ DELMIA $\bullet$ Python $\bullet$ {TIA Portal V14/V15} $\bullet$ Ladder $\bullet$ Function Block $\bullet$ SCL/2},{MATLAB $\bullet$ Simulink $\bullet$ AMESim/3},{Office (Word, Excel, PowerPoint) $\bullet$ {LibreOffice} $\bullet$ Windows/5}}

\skillstext{{Oprogramowanie/ },{fsdf/ }}
\startext{\textbf{Oprogramowanie biurowe:} /4,_,{Office$($/4},{Word/4},{Excel/4},{Power Point$)$,}/3,{LibreOffice,}/4,Windows/4,_, 
\textbf{Symulacje:}/,_,AMESim/3,Simulink/3,DELMIA/1,_, 
\textbf{Programowanie PLC:}/,_, Siemens:/2, {TIA Portal (LD, FB, SCL),}/,AllenBradley:/1, {Studio5000 (LD, ST)}/,_, 
\textbf{Programowanie skryptowe:}/,_,{MS Visual Studio$($}/2, {C/C++}/2, {Python$)$,}/3, Matlab/2}
%------------------------------------------------

% Skill text section, each skill must have a value between 0 an 6
%\skillstext{{skill_1/skilllevel_1},{skill_1/skilllevel_2}}

\languages{{Język angielski/{B2/C1}}, {Język niemiecki/A1}}

\certificates{ {Prawo jazdy kat. B / - - -}, {SEP E1 do 1kV  / 02.2024}}

%----------------------------------------------------------------------------------------

\makeprofile % Print the sidebar
\hyphenpenalty=10000  %brak dzielenia wyrazów
%----------------------------------------------------------------------------------------
%	 INTERESTS
%----------------------------------------------------------------------------------------

\section{Doświadczenie zawodowe}

\begin{twenty} % Environment for a list with descriptions
	\twentyitem{obecnie- -07.2018 \newline (1~rok 2~mies.)\newline}{Inżynier ds. Projektowania \newline i Programowania}{Primetals Technologies Poland}{
    \begin{itemize}
				\item Programowanie sterowników PLC (\textbf{TIA Portal, Studio5000})
				\item Testowanie stanowisk, szaf, urządzeń - I/O check (\textbf{TIA Portal, Studio5000})
				\item Uruchamianie i testowanie urządzeń (\textbf{TIA Portal, Studio5000})

			\end{itemize}}
	
	\twentyitem{09.2016- -03.2014 \newline (2~lata 6~mies.)}{Uczestnik projektu}{Politechnika Wrocławska}{Projekt \quotedblbase\emph{Opracowanie innowacyjnych rozwiązań	wysokociśnieniowych pomp łopatkowych ze zintegrowanym	mechatronicznym napędem elektrycznym}\textquotedblright \ wykonywanego na zamówienie 	\emph{Narodowego Centrum Badań i Rozwoju}
    \begin{itemize}
				\item Przeprowadzenie obliczeń symulacyjnych przebiegu momentu obciążającego w~pompie łopatkowej (\textbf{AMESim, MATLAB-Simulink})
				\item Opracowanie poprawek technologicznych dla pompy łopatkowej przeznaczonej do zabudowy w~silniku elektrycznym (\textbf{AMESim, MATLAB-Simulink})
				\item Opracowanie modelu symulacji przebiegów ciśnienia w~pompie eksperymentalnej (\textbf{AMESim, MATLAB-Simulink})
				\item Opracowanie modelu symulacji przebiegów momentu i~przebiegów ciśnienia przy stanach ustalonych i~dynamicznych (\textbf{AMESim, MATLAB-Simulink})
				\item Opracowanie wyników pomiarów hydraulicznych zespołu silnik-pompa (\textbf{AMESim, MATLAB-Simulink})
			\end{itemize}}
			
		
	\twentyitem{02.2015- -06.2014 \newline (8 mies.)}{Inżynier robotyki}{RW Swiss Automation}{
	\begin{itemize}
				\item Programowanie offline robotów przemysłowych (\textbf{DELMIA})
				\item Symulacja procesu (\textbf{DELMIA})
				\item Sporządzanie dokumentacji stanowisk (\textbf{DELMIA,MS Office: Word, Excel, PowerPoint})
			\end{itemize} 
			}
\end{twenty}





%----------------------------------------------------------------------------------------
%	 EDUCATION
%----------------------------------------------------------------------------------------

\section{Wykształcenie}

\begin{twenty} % Environment for a list with descriptions
	\twentyitem{obecnie- -10.2014}{{Budowa i Eksploatacja Maszyn}}{{Wydział~Mechaniczny} {Politechnika~Wrocławska}}{\vspace{-\baselineskip}{{studia doktoranckie}\\{\emph{temat przewodu:}
\quotedblbase{Analiza dynamiki łopatek w pompie łopatkowej pojedynczego działania}\textquotedblright}}}
	
	\twentyitem{06.2014- -02.2012}{{Automatyka i Robotyka }(Mgr inż.)}{{Wydział~Mechaniczny} {Politechnika~Wrocławska}}{{\emph{specjalność:} Automatyzacja Maszyn i Procesów Roboczych}\\{\emph{temat pracy:}
\quotedblbase{Opracowanie układu sterowania dla mechatronicznej pompy wyporowej}\textquotedblright}}
	
	\twentyitem{01.2012- -10.2008}{{Automatyka i Robotyka }(Inż.)}{{Wydział~Mechaniczny} {Politechnika~Wrocławska}}{{{\emph{temat pracy:} \quotedblbase{Opracowanie technologii spawania podzespołu fotela samochodowego 	na zrobotyzowanym stanowisku}\textquotedblright}}}
	
	%\twentyitem{<dates>}{<title>}{<location>}{<description>}
\end{twenty}

%----------------------------------------------------------------------------------------
%	 PUBLICATIONS
%----------------------------------------------------------------------------------------

\section{Szkolenia}

\begin{twentyshort} % Environment for a short list with no descriptions
	\twentyitemshort{10.2015}{\quotedblbase\emph{Entrepreneurship and Soft Skills Training Program for PhDs and Young Scientists}\textquotedblright \ {w Alberta School of Business, University of Alberta, Kanada}}
	%\twentyitemshort{<dates>}{<title/description>}
\end{twentyshort}

%----------------------------------------------------------------------------------------
%	 FOOTNOTE
%----------------------------------------------------------------------------------------
\blfootnote{    Wyrażam zgodę na przetwarzanie moich danych osobowych dla potrzeb niezbędnych do realizacji procesu rekrutacji (zgodnie z Ustawą z dnia 29.08.1997 roku o Ochronie Danych Osobowych; tekst jednolity: Dz. U. 2016 r. poz. 922).}


%----------------------------------------------------------------------------------------
%	 OTHER INFORMATION
%----------------------------------------------------------------------------------------

%\section{Other information}

%\subsection{Review}

%----------------------------------------------------------------------------------------
%	 SECOND PAGE EXAMPLE
%----------------------------------------------------------------------------------------

%\newpage % Start a new page

%\makeprofile % Print the sidebar

%\section{Other information}

%\subsection{Review}

%Alice 

%----------------------------------------------------------------------------------------

\end{document} 
